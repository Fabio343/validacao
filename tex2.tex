\documentclass[a4paper, 12pt]{article}
\usepackage[top=2cm, bottom=2cm, left=2.5cm, right=2.5cm]{geometry}
\usepackage[utf8]{inputenc}
\usepackage{amsmath, amsfonts, amssymb}
\DeclareMathOperator{\sen}{sen}


\begin{document}
 
\begin{enumerate}
\item Notação de Matrizes
\begin{enumerate}
 \item Considere a matriz
 $M=\begin{bmatrix}
 1 & 5 & 5 \\
 5 & 8 & 6
 \end{bmatrix}
 $
 \item Calcule o determinante e  o que se pede abaixo
  $\begin{vmatrix}
 1 & 5 & 5 \\
 5 & 8 & 6 \\
 6 & 8 & 7
 \end{vmatrix}
 $
 \begin{enumerate}

 \item $M^{-1}$
 \item $M^{T}$
 \end{enumerate}
 
 \item Considere $m\times n$
 $\begin{bmatrix}
 a_{11} & a_{12} & a_{13} & \cdots & a_{1n}\\
 a_{21} & a_{22} & a_{23} & \cdots & a_{2n}\\
 a_{31} & a_{32} & a_{33} & \cdots & a_{3n}\\
 \vdots & \vdots & \vdots & \vdots & \vdots\\
 a_{m1} & a_{m2} & a_{m3} & \cdots & a_{mn}\\
 \end{bmatrix}$
 
 \item Determine $x$, $y$, $z$ na equação
 $$\begin{bmatrix}
 1 & 5 & 5 \\
 5 & 8 & 6 \\
 5 & 8 & 6
 \end{bmatrix}
 \begin{bmatrix}
  x \\
  y \\
  z \\
 \end{bmatrix}
 =
 \begin{bmatrix}
 5 \\
 6 \\
 8 \\
 \end{bmatrix}
 $$
\end{enumerate}

 \item Notação de funções
 \begin{enumerate}
 \item Seja $f : \mathbb{R} \to \mathbb{R}$ definida por $f(x) = \dfrac{1}{2}x^2 - 2x + 10$
 \item $\\f(x) =
 \begin{cases}
 x^2 -1 ;\,x \geq 1\\
 2x+1;\,x <1
 \end{cases}$
 
 \item $\\f(x) =
 \begin{cases}
 x^2 -1 ;\, \textrm{se }x \geq 1\\
 2x+1;\, \textrm{se }x <1
 \end{cases}$
 
 \item $f(x)= \log_2 x$
 \item $f(x)= \cos x$
 \item $f(x)= \sin x$
 \item $f(x)=\sen x$
 \item $f(x)=\sen \left(x-\frac{\pi}{2}\right)$
 \end{enumerate}
 \item Listas e Operações Básicas
 \begin{enumerate}
 \item $a + b $
 \item $a - b $
 \item $a\cdot b $
 \item $a \div b $
 \item $\dfrac{a}{b} $
 \item $\sqrt[3]{x} $
 \item $\sqrt{y}$
 \item $ a^{x*y} $
 \end{enumerate}
 \item Notações de Conjuntos
 \begin{enumerate}
 \item $ A = \{1;\ 2;\ 3;\}$
 \item $B = \{x \in \mathbb{Z} \, |\, -2 \leq x<5 \}$
 \item $A \cap B$
 \item $B \cup C$
 \item $B - C $
 \item $C\setminus B $
 \item $\mathbb{Z} \supset \mathbb{N}$
 \item $\mathbb{Z} \not\supset \mathbb{N}$
 \item $\forall x \in \mathbb{N}$, temos $x \geq 0$
 \item $\exists x \in \mathbb{R}$, tal que $\sqrt{x}\not\in\mathbb{R}$
 \end{enumerate}
\end{enumerate}
\end{document}